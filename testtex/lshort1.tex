\documentclass[10pt,a4paper]{article}
\usepackage[latin1]{inputenc}
\usepackage{amsmath}
\usepackage{amsfonts}
\usepackage{amssymb}
\author{Kevin Q.}
\title{A test article for lshort}
\begin{document}

Add $a$ squared and $b$ squared
to get $c$ squared. Or, using
a more mathematical approach:
$c^{2}=a^{2}+b^{2}$

\TeX{} is pronounced as
$\tau\epsilon\chi$.\\[6pt]
100~m$^{3}$ of water\\[6pt]
This comes from my $\heartsuit$`

Add $a$ squared and $b$ squared
to get $c$ squared. Or, using
a more mathematical approach:
\begin{displaymath}
c^{2}=a^{2}+b^{2}
\end{displaymath}
And just one more line.

\begin{equation} \label{eq:eps}
\epsilon > 0
\end{equation}
From (\ref{eq:eps}), we gather
\ldots

\begin{displaymath}
\lim_{n \to \infty}
\sum_{k=1}^n \frac{1}{k^2}
= \frac{\pi^2}{6}
\end{displaymath}

\begin{displaymath}
x^{2} \geq 0\qquad
\textrm{for all }x\in\mathbb{R}
\end{displaymath}

$\lambda,\xi,\pi,\mu,\Phi,\Omega$

\newcommand{\ud}{\mathrm{d}}
\begin{displaymath}
\int\!\!\!\int_{D} g(x,y)
\, \ud x\, \ud y
\end{displaymath}
instead of
\begin{displaymath}
\int\int_{D} g(x,y)\ud x \ud y
\end{displaymath}

\begin{displaymath}
\mathbf{X} =
\left( \begin{array}{ccc}
x_{11} & x_{12} & \ldots \\
x_{21} & x_{22} & \ldots \\
\vdots & \vdots & \ddots
\end{array} \right)
\end{displaymath}

\begin{displaymath}
\mathop{\mathrm{corr}}(X,Y)=
\frac{\displaystyle
\sum_{i=1}^n(x_i-\overline x)
(y_i-\overline y)}
{\displaystyle\biggl[
\sum_{i=1}^n(x_i-\overline x)^2
\sum_{i=1}^n(y_i-\overline y)^2
\biggr]^{1/2}}
\end{displaymath}

Partl~\cite{pa} has
proposed that \ldots

\begin{thebibliography}{99}
\bibitem{pa} H.~Partl:
\emph{German \TeX},
TUGboat Volume~9, Issue~1 (1988)
\bibitem{pa1} H.~Partl:
\emph{German \TeX},
TUGboat Volume~9, Issue~1 (1988)
\end{thebibliography}

\end{document}